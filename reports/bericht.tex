\documentclass[a4paper,ngerman]{article}
% generated by Docutils <http://docutils.sourceforge.net/>
\usepackage{fixltx2e} % LaTeX patches, \textsubscript
\usepackage{cmap} % fix search and cut-and-paste in Acrobat
\usepackage{ifthen}
\usepackage[T1]{fontenc}
\usepackage[utf8]{inputenc}
\usepackage[ngerman]{babel}
\usepackage{color}
\usepackage{float} % float configuration
\floatplacement{figure}{H} % place figures here definitely
\usepackage{graphicx}
\setcounter{secnumdepth}{0}

%%% Custom LaTeX preamble
% PDF Standard Fonts
\usepackage{mathptmx} % Times
\usepackage[scaled=.90]{helvet}
\usepackage{courier}

%%% User specified packages and stylesheets
\usepackage{etc/pygments-docutilsroles}

%%% Fallback definitions for Docutils-specific commands
% basic code highlight:
\providecommand*\DUrolecomment[1]{\textcolor[rgb]{0.40,0.40,0.40}{#1}}
\providecommand*\DUroledeleted[1]{\textcolor[rgb]{0.40,0.40,0.40}{#1}}
\providecommand*\DUrolekeyword[1]{\textbf{#1}}
\providecommand*\DUrolestring[1]{\textit{#1}}

% inline markup (custom roles)
% \DUrole{#1}{#2} tries \DUrole#1{#2}
\providecommand*{\DUrole}[2]{%
  \ifcsname DUrole#1\endcsname%
    \csname DUrole#1\endcsname{#2}%
  \else% backwards compatibility: try \docutilsrole#1{#2}
    \ifcsname docutilsrole#1\endcsname%
      \csname docutilsrole#1\endcsname{#2}%
    \else%
      #2%
    \fi%
  \fi%
}

% hyperlinks:
\ifthenelse{\isundefined{\hypersetup}}{
  \usepackage[colorlinks=true,linkcolor=blue,urlcolor=blue]{hyperref}
  \urlstyle{same} % normal text font (alternatives: tt, rm, sf)
}{}


%%% Body
\begin{document}






















\ \ 
\hfill
\begin{minipage}[t]{5cm}
\includegraphics[width=5cm]{img/hsa-logo.jpg}
\end{minipage}

\vskip 10mm

{\parindent=0pt

\thispagestyle{empty}


{\Large\bf Praktikum Embedded Linux
}

\vskip 5mm

Embedded Linux \\
Hochschule Augsburg \\
Fakultät für Informatik (Prof. Dr. Hubert Högl) \\
Studiengang Technische Informatik, 4. Semester \\

Sommersemester 2016
 

Datum: 2016-07-05 18:04 \\

\medskip

\rule{10cm}{4pt}\\

\medskip

Philip Manke, $<$Philip.Manke@hs-augsburg.de$>$, \#945072 \\


\vskip 2cm

\begin{center}
{\LARGE\bf

Schlafüberwachung mit dem Beaglebone Black

}
\end{center}

\vfill

\begin{minipage}[t]{3cm}
\includegraphics[width=3cm]{img/cc-logo.jpg}
\end{minipage}

{\small
Dieser Text steht unter der Creative Commons Lizenz "Namensnennung/Keine kommerizelle Nutzung"\\
http://creativecommons.org/licenses/by-nc/3.0/de/
}

} % parindent


\newpage


\phantomsection\label{inhalt}
\pdfbookmark[1]{Inhalt}{inhalt}
\renewcommand{\contentsname}{Inhalt}
\tableofcontents



\section{1~~~Einleitung%
  \label{einleitung}%
}

Dieser Bericht bezieht sich auf den Versuch V2 \textquotedbl{}Das Arduino Projekt\textquotedbl{}.

Es sind keine expliziten Versuchsanweisungen gegeben, außer di

Sie müssen in Ihrem Studium öfter Texte wie zum Beispiel Praktikums- und
Versuchsberichte abgeben. Als langjähriger \textquotedbl{}Empfänger\textquotedbl{} von diesen Texten
habe ich mir mal überlegt, wie ein \textquotedbl{}idealer\textquotedbl{} Bericht sowohl aus Ihrer, als
auch aus meiner Sicht aussehen könnte:

Studentische Sicht:
%
\begin{quote}
%
\begin{itemize}

\item Einfach zu schreiben in einem gewöhnlichen (Programmier-)Editor.

\item Die Titelseite wird automatisch erstellt.

\item Ein einheitliches Format kann in HTML und PDF umgewandelt werden.

\item Funktioniert mit Linux und Windows

\end{itemize}

\end{quote}

Aus meiner Sicht:
%
\begin{quote}
%
\begin{itemize}

\item Einheitliches Aussehen

\item Gut in Web-Dokument umwandelbar

\item Format einfach automatisiert weiterverarbeitbar, z.B. in gedruckte
Berichtsammlungen

\item Kleine Dateien

\end{itemize}

\end{quote}

Dieses einheitliche Format gibt es bereits, es nennt sich \textquotedbl{}reStructuredText\textquotedbl{}
(reST).  Es entstand vor einigen Jahren in der Welt der Programmiersprache
Python. Die gesamte Python Dokumentation wird mittlerweile in diesem Format
geschrieben und mit \cite{SPHINX} automatisch in die Darstellungen Text, HTML und
PDF umgewandelt, siehe \cite{PYDOC}. Die reST Homepage ist \cite{REST}.


\section{2~~~Vorbereiten des Rechners%
  \label{vorbereiten-des-rechners}%
}
%
\begin{quote}
%
\begin{description}
\item[{Windows}] \leavevmode 
Sie brauchen den Python Interpreter \cite{PYTHON} und das Docutils Paket
\cite{DOCUTILS}.  Das reicht für die Ausgabe von HTML Dateien. Falls Sie
ausserdem Dokumente im PDF Format erzeugen wollen, müssen Sie \emph{TeX}
installiere, ich empfehle \cite{MIKTEX} oder \cite{TEXLIVE}.

\item[{Linux}] \leavevmode 
Sie brauchen den Python Interpreter \cite{PYTHON} und das Docutils Paket,
siehe \cite{DOCUTILS}.  Das reicht für die Ausgabe von HTML Dateien. Falls Sie
ausserdem Dokumente im PDF Format erzeugen wollen, müssen Sie \emph{TeX}
installieren, ich empfehle \cite{TEXLIVE}. Sie sollten auch noch diese
Zusatzpakete für Latex installieren: \texttt{texlive-latex-extra} und
\texttt{texlive-lang-french}.

\end{description}

\end{quote}


\section{3~~~So gehen Sie vor%
  \label{so-gehen-sie-vor}%
}

In diesem \href{http://elk.informatik.fh-augsburg.de/pub/Demo-Bericht/}{Verzeichnis} finden Sie das
aktuelle ZIP Archiv des Demo-Berichtes.  Sie holen und entpacken es, dann
sollten folgende Dateien in dem entpackten Unterverzeichnis \texttt{Demo-Bericht}
zu finden sein:
%
\begin{quote}{\ttfamily \raggedright \noindent
bericht.cfg~~bericht.rst~~etc/~~Makefile~VERSION
}
\end{quote}

Den Bericht tippen Sie in \texttt{bericht.rst}. Sie können alle Markup Anweisungen
von Restructured Text verwenden, eine gute Schnellreferenz finden Sie in
\cite{QUICKREF}.

In \texttt{bericht.cfg} schreiben Sie den Titel des Berichtes, alle Namen mit
Email-Adressen und Matrikelnummer der Projektgruppe. Auch das aktuelle Semester
sollten Sie hier eintragen.

\textbf{Linux}
%
\begin{quote}

Sie können den Bericht durch ein Makefile sowohl im HTML- als auch
im PDF-Format ausgeben. Geben Sie dazu das Kommando \texttt{make html} oder \texttt{make
pdf} ein.   Damit dies funktioniert muss auf Ihrem Rechner das Paket
\texttt{python-docutils} installiert sein. Darin gibt es unter anderem die Skripte
\texttt{rst2html} und \texttt{rst2latex}, die das Makefile verwendet.

Sie können alle generierten Dateien wieder löschen mit dem Kommando \texttt{make
clean}. Sollten die erzeugten Ausgabeformate nicht das enthalten, was Sie
wollen, dann kann oft ein \texttt{make clean} \emph{vor} \texttt{make html} oder \texttt{make
pdf} helfen.

\end{quote}

\textbf{Windows}
%
\begin{quote}

Sie haben zwei Möglichkeiten:
\newcounter{listcnt0}
\begin{list}{\arabic{listcnt0}.}
{
\usecounter{listcnt0}
\setlength{\rightmargin}{\leftmargin}
}

\item Installieren Sie die Unix Utilities und verwenden Sie wie auf Linux
\texttt{make}.

\url{http://unxutils.sourceforge.net}

\item Rufen Sie die Kommandos von Hand auf (oder schreiben Sie eine
Batch Datei):

HTML:
%
\begin{quote}{\ttfamily \raggedright \noindent
rst2html~-{}-language=de~-{}-stylesheet=etc/goodger.css~\textbackslash{}\\
~~~bericht.rst~bericht.html
}
\end{quote}

PDF:
%
\begin{quote}{\ttfamily \raggedright \noindent
rst2latex~-{}-language=de~\textbackslash{}\\
~~~-{}-stylesheet=etc/pygments-docutilsroles.sty~\textbackslash{}\\
~~~bericht.rst~bericht.tex\\
pdflatex~bericht.tex\\
pdflatex~bericht.tex
}
\end{quote}

Sie müssen zweimal \texttt{pdflatex} laufen lassen, damit das
Inhaltsverzeichnis erstellt wird.
\end{list}

\end{quote}


\section{4~~~So funktioniert es%
  \label{so-funktioniert-es}%
}

Damit das Deckblatt des Berichtes einheitlich aussieht, müssen Sie es gar nicht
selber schreiben, sondern es wird vom Computer automatisch erstellt. Sie müssen
nur ein paar Daten in die Datei \url{bericht.cfg} eingeben.  Die eingegebenen
Daten sind in einem bestimmten Format, das von \textquotedbl{}empy\textquotedbl{} -{}- einem einfachen, in
Python geschriebenen Template Programm -{}- verwendet wird.  Sie finden empy in
\url{etc/em.py}. Empy ersetzt in \url{etc/kopf.rst.in},
\url{etc/titelseite.html.in} und \url{etc/titelseite.tex.in} die Templates durch
den tatsächlichen Wert, der aus der Konfigurationsdatei genommen wird und
erstellt dann die Dateien \emph{ohne} die Endung \texttt{.in}. Die Homepage von Empy
ist \cite{EMPY}.


\section{5~~~Die Lizenz%
  \label{die-lizenz}%
}

Ich schlage vor, dem Bericht eine einheitliche Lizenz zu geben und zwar die
Creative Commons Lizenz \url{http://creativecommons.org/licenses/by-nc/3.0/de/}.
Diese Lizenz finden Sie voreingestellt auf jeder Titelseite. Wenn Sie als
Autorin und Autor nicht zustimmen, dann dürfen Sie gerne eine andere Lizenz
wählen.  Schreiben Sie mir eine E-mail an <\href{mailto:Hubert.Hoegl@hs-augsburg.de}{Hubert.Hoegl@hs-augsburg.de}> falls
Sie eine andere Lizenz wollen.


\section{6~~~Ein paar Fomatierungshinweise%
  \label{ein-paar-fomatierungshinweise}%
}

Die \cite{QUICKREF} zeigt Ihnen schnell die wichtigsten allgemeinen
Formatierungsbefehle, wie Schriftstile, Überschriften, Listen, Aufzählungen und
so weiter.

Da Sie im Praktikum häufig mit Quelltext zu tun haben werden, zeige ich Ihnen,
wie man Quelltext einbindet, so dass die Syntax hervorgehoben wird
(zumindest in der HTML Ausgabe).  Hier ist ein Beispiel für C Code:
%
\begin{quote}{\ttfamily \raggedright \noindent
\DUrole{ln}{1~}\DUrole{keyword}{\DUrole{type}{int}}~\DUrole{name}{\DUrole{function}{main}}\DUrole{punctuation}{()}~\\
\DUrole{ln}{2~}\DUrole{punctuation}{\{}~\\
\DUrole{ln}{3~}~~~\DUrole{name}{printf}\DUrole{punctuation}{(}\DUrole{literal}{\DUrole{string}{\textquotedbl{}Hello~World}}\DUrole{literal}{\DUrole{string}{\DUrole{escape}{\textbackslash{}n}}}\DUrole{literal}{\DUrole{string}{\textquotedbl{}}}\DUrole{punctuation}{);}~\\
\DUrole{ln}{4~}~~~\DUrole{name}{exit}~\DUrole{literal}{\DUrole{number}{\DUrole{integer}{0}}}\DUrole{punctuation}{;}~\\
\DUrole{ln}{5~}\DUrole{punctuation}{\}}
}
\end{quote}

Der reST Quelltext sieht dazu so aus:
%
\begin{quote}{\ttfamily \raggedright \noindent
..~code::~c\\
~~~:number-lines:~1\\
~\\
~~~int~main()\\
~~~\{\\
~~~~~~printf(\textquotedbl{}Hello~World\textbackslash{}n\textquotedbl{});\\
~~~~~~exit~0;\\
~~~\}
}
\end{quote}

Das kleine \texttt{c} hinter \texttt{code::} gibt die Sprache an, es gibt für fast
jede Sprache ein Kürzel, siehe \url{http://pygments.org/languages}.

\textbf{Hinweis:} Damit die Syntax bei eingebundenem Quelltext hervorgehoben wird,
muss man eine relativ moderne Docutils Version verwenden.  Bei Versionen
\emph{kleiner als} 0.9.1 klappt es nicht! So findet man die Version heraus:
%
\begin{quote}{\ttfamily \raggedright \noindent
\$~rst2html~-{}-version\\
rst2html~(Docutils~0.9.1~{[}release{]},~Python~2.6.5,~on~linux2)
}
\end{quote}

Sie können Code auch aus einer Datei importieren, das geht so:
%
\begin{quote}{\ttfamily \raggedright \noindent
..~include::~etc/demo.c\\
~~~:code:~c
}
\end{quote}

So sieht das Ergebnis dann aus:
%
\begin{quote}{\ttfamily \raggedright \noindent
\DUrole{keyword}{\DUrole{type}{int}}~\DUrole{name}{\DUrole{function}{main}}\DUrole{punctuation}{()}~\\
\DUrole{punctuation}{\{}~\\
~~~\DUrole{keyword}{\DUrole{type}{int}}~\DUrole{name}{i}\DUrole{punctuation}{;}~\\
~~~\DUrole{keyword}{for}~\DUrole{punctuation}{(}\DUrole{name}{i}\DUrole{operator}{=}\DUrole{literal}{\DUrole{number}{\DUrole{integer}{0}}}\DUrole{punctuation}{;}~\DUrole{name}{i}\DUrole{operator}{<}\DUrole{literal}{\DUrole{number}{\DUrole{integer}{10}}}\DUrole{punctuation}{;}~\DUrole{name}{i}\DUrole{operator}{++}\DUrole{punctuation}{)}~\DUrole{punctuation}{\{}~\\
~~~~~~\DUrole{name}{printf}\DUrole{punctuation}{(}\DUrole{literal}{\DUrole{string}{\textquotedbl{}\%d}}\DUrole{literal}{\DUrole{string}{\DUrole{escape}{\textbackslash{}n}}}\DUrole{literal}{\DUrole{string}{\textquotedbl{}}}\DUrole{punctuation}{,}~\DUrole{name}{i}\DUrole{punctuation}{);}~\\
~~~\DUrole{punctuation}{\}}~\\
~~~\DUrole{name}{exit}~\DUrole{literal}{\DUrole{number}{\DUrole{integer}{0}}}\DUrole{punctuation}{;}~\\
\DUrole{punctuation}{\}}
}
\end{quote}

Abbildungen bauen Sie wie folgt ein:
%
\begin{quote}{\ttfamily \raggedright \noindent
..~figure::~img/python-powered-w-200x80.png\\
~~~:align:~center\\
~\\
~~~Hier~sind~alle~`Python~Logos\\
~~~<http://www.python.org/community/logos/>`\_.
}
\end{quote}

So sieht das Ergebnis aus:

\begin{figure}
\noindent\makebox[\textwidth][c]{\includegraphics{img/python-powered-w-200x80.png}}
\caption{Hier sind alle \href{http://www.python.org/community/logos/}{Python Logos}.}
\end{figure}


\section{7~~~Mögliche Verbesserungen%
  \label{mogliche-verbesserungen}%
}
%
\begin{itemize}

\item Man könnte das Programm anpassbar auf andere Veranstaltungen machen. Das
heisst, der Name der Veranstaltung (Versuche aus der Technischen Informatik)
und die Person, bei der abzugeben ist, könnten konfigurierbar sein.
Ich würde mich freuen, wenn Studenten dieses Ideen in die Tat umsetzen würden

\item Wahrscheinlich macht mein Ansatz mit \texttt{make} Probleme unter Windows.
Hier sollte noch eine Lösung gefunden werden, wie man unter Windows
bequem mit der Bericht-Umgebung arbeiten kann.

\end{itemize}

Ich freue mich auf Ihre Nachricht an <\href{mailto:Hubert.Hoegl@hs-augsburg.de}{Hubert.Hoegl@hs-augsburg.de}>.


\section{8~~~Literatur%
  \label{literatur}%
}

% vim: et sw=4 ts=4

\begin{thebibliography}{DOCUTILS}
\bibitem[DOCUTILS]{DOCUTILS}{
Docutils Paket

\url{http://docutils.sourceforge.net}
}
\bibitem[REST]{REST}{
reST Homepage

\url{http://docutils.sourceforge.net/rst.html}
}
\bibitem[QUICKREF]{QUICKREF}{
reST Quick Reference

\url{http://docutils.sourceforge.net/docs/user/rst/quickref.html}
}
\bibitem[PYTHON]{PYTHON}{
Python Interpreter

\url{http://www.python.org}
}
\bibitem[PYDOC]{PYDOC}{
Python Standard Dokumentation

\url{http://www.python.org/doc}
}
\bibitem[SPHINX]{SPHINX}{
Python Documentation Generator

\url{http://sphinx.pocoo.org}
}
\bibitem[MIKTEX]{MIKTEX}{
TeX und LaTeX für Windows

\url{http://miktex.org}
}
\bibitem[TEXLIVE]{TEXLIVE}{
TeX Live

\url{http://www.tug.org/texlive}
}
\bibitem[EMPY]{EMPY}{
Empy Macro Processor

\url{http://www.alcyone.com/software/empy}
}
\end{thebibliography}

\end{document}
